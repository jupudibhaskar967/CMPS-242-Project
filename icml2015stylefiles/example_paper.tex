%%%%%%%%%%%%%%%%%%%%%%%%%%%%%%%%%%%%%%%%%%%%%%%%%%%%%%%%%%%%%%%%%%
%%%%%%%% ICML 2015 EXAMPLE LATEX SUBMISSION FILE %%%%%%%%%%%%%%%%%
%%%%%%%%%%%%%%%%%%%%%%%%%%%%%%%%%%%%%%%%%%%%%%%%%%%%%%%%%%%%%%%%%%

% Use the following line _only_ if you're still using LaTeX 2.09.
%\documentstyle[icml2015,epsf,natbib]{article}
% If you rely on Latex2e packages, like most moden people use this:
\documentclass{article}

% use Times
\usepackage{times}
% For figures
\usepackage{graphicx} % more modern
%\usepackage{epsfig} % less modern
\usepackage{subfigure} 

% For citations
\usepackage{natbib}

% For algorithms
\usepackage{algorithm}
\usepackage{algorithmic}

% As of 2011, we use the hyperref package to produce hyperlinks in the
% resulting PDF.  If this breaks your system, please commend out the
% following usepackage line and replace \usepackage{icml2015} with
% \usepackage[nohyperref]{icml2015} above.
\usepackage{hyperref}

% Packages hyperref and algorithmic misbehave sometimes.  We can fix
% this with the following command.
\newcommand{\theHalgorithm}{\arabic{algorithm}}

% Employ the following version of the ``usepackage'' statement for
% submitting the draft version of the paper for review.  This will set
% the note in the first column to ``Under review.  Do not distribute.''
\usepackage{icml2015} 

% Employ this version of the ``usepackage'' statement after the paper has
% been accepted, when creating the final version.  This will set the
% note in the first column to ``Proceedings of the...''
%\usepackage[accepted]{icml2015}


% The \icmltitle you define below is probably too long as a header.
% Therefore, a short form for the running title is supplied here:
\icmltitlerunning{Submission and Formatting Instructions for ICML 2015}

\begin{document} 
\twocolumn[
\icmltitle{Sentimental analysis for yelp dataset}

% It is OKAY to include author information, even for blind
% submissions: the style file will automatically remove it for you
% unless you've provided the [accepted] option to the icml2015
% package.

% You may provide any keywords that you 
% find helpful for describing your paper; these are used to populate 
% the "keywords" metadata in the PDF but will not be shown in the document
\icmlkeywords{boring formatting information, machine learning, ICML}
%%\vskip 0.3in
]

\icmlauthor[1]{Bhaskar Jupudi}{njupudi@ucsc.edu}
\icmlauthor[2]{Trivikram Bollempalli}{tbollemp@ucsc.edu}

\begin{abstract} 
In this project, we aim to perform sentiment analysis i.e., classifying whether the review is postive or negative using the yelp dataset based on reviews and ratings. The classification problem can be solved by a set of algorithms. Every algorithm has its own advantages and disadvantages in terms of accuracy and model complexity. 

For example, Naive Bayes classifier is faster to compute than Logistic Regression classifier for huge datasets. But the disadvantage with the former classifier is that it assumes that features are independent where as the latter has no such assumptions which can lead to better precdiction. Our work mainly concentrates on implementing these two classifiers and techniques to make them perform much better. We have adopted multi-processing for feature extraction to make it way faster and also implemented two different approaches of Logistic regression for both binary and multi-class classfication. We have also implemented Naive Bayes classifier. Finally, we contrast these two algorithms based on time taken for execution and performance metrics like accuracy, precision and recall. 
\end{abstract} 

\section{Problem statement}
asasd

\section{Feature Extraction}
asdasdas
\section{Model Formulation}
\section{Evaluations}
\section{Results}








% In the unusual situation where you want a paper to appear in the
% references without citing it in the main text, use \nocite
\nocite{langley00}

\bibliography{example_paper}
\bibliographystyle{icml2015}

\end{document} 


% This document was modified from the file originally made available by
% Pat Langley and Andrea Danyluk for ICML-2K. This version was
% created by Lise Getoor and Tobias Scheffer, it was slightly modified  
% from the 2010 version by Thorsten Joachims & Johannes Fuernkranz, 
% slightly modified from the 2009 version by Kiri Wagstaff and 
% Sam Roweis's 2008 version, which is slightly modified from 
% Prasad Tadepalli's 2007 version which is a lightly 
% changed version of the previous year's version by Andrew Moore, 
% which was in turn edited from those of Kristian Kersting and 
% Codrina Lauth. Alex Smola contributed to the algorithmic style files.  
\grid
